% intro talk in german
%\documentclass[NET,a4paper,12pt,ngerman]{netforms}

% intro talk in english
\documentclass[NET,a4paper,12pt,english]{netforms}

\usepackage[utf8]{inputenc}
\usepackage{tumlang}
\usepackage{tumcontact}
\usepackage{scrlayer-scrpage}
\usepackage[textsize=scriptsize]{todonotes}
% Disable todonotes with
%\setuptodonotes{disable}
\setuptodonotes{inline}
\usepackage{pgfgantt}
\usepackage{float}
\usepackage{enumitem}
\newlist{researchquestions}{enumerate}{1}
\setlist[researchquestions,1]{label=\textbf{\textit{RQ \arabic*}}, resume}


\geometry{%
	top=20mm,
	bottom=20mm,
	left=25mm,
	right=25mm,
	headsep=1.5cm,
	includehead,
}

% Alle Konfigurationsbefehle sind optional. Fehlende Befehle fueheren einfach
% zu "blank forms".

% Type of thesis
% valid parameters: bachelor,master,diplom,idp,gr,hiwi,other
% If 'other' is chosen an optional parameter can be passed along (\type[optional]{other}).
% Without an optional parameter 'other' is chosen as a description.
\type{idp}

% \studiengang{} details the field of study
% valid parameters:
%   "Informatik",
%   "Wirtschaftsinformatik",
%   "Robotics, Cognition, Intelligence"
%   "Informatik: Games Engineering"
\studiengang{Informatik}

% Informationen ueber den Studenten. Sollte selbsterklaerend sein.
\anrede{Herr}
\nachname{Castellotti}
\vorname{Roberto}
\matrikel{03767095}
\rbgaccount{casr}
\semester{2}{SoSe\,2023}
\studientelefon{}{0039379181686}
\heimattelefon{}{--}
\studienadresse{Kantsraße}{17 München}
\heimatadresse[adresszusatz=,appartment=]{Kantsraße}{17}
\mail{r.castellotti@tum.de}

% Informationen ueber die Arbeit. Sollte selbsterklaerend sein.
\themensteller{\chairhead}
\beginn{03}{2023}
\endt{08}{2023}
\betreuer{Leander Seidlitz, Johannes Zirngibl}
\title{Development of a Framework for Retrieval of Parameters of the Starlink Dishy}{Entwurf eines Frameworks zur Informationsgewinnung von Parametern der Starlink Dishy}

% Nur für Bachelorarbeiten
\sprache{englisch} % options: deutsch | englisch

% Falls \type{hiwi} gesetzt wurde, wird die Taetigkeit auf dem Aufnahmeformular
% des Lehrstuhls angegeben.
\taetigkeit{test}



\pagestyle{scrheadings}
\clearscrheadfoot
\chead{\TUMheader{1cm}}

\renewcommand{\maketitle}{%
	\begin{center}
		\textbf{\introductoryheadline}%

		\Large%
		\textbf{\thetitle}%
	\end{center}

	\footnotesize%
	\hrule
	\vskip1ex
	\begin{tabular}{ll}
		\thenamelabel: & \thevorname{} \textbf{\thenachname}\\
		\theadvisorlabel: & \hspace*{-.5ex}\thebetreuer\\
		\thesupervisorlabel: & \chairhead\\
		\thebeginlabel: & \thebeginnmonat/\thebeginnjahr\\
		\theendlabel: & \theendmonat/\theendjahr\\
	\end{tabular}
	\vskip1ex
	\hrule
	\vskip4ex
}

\linespread{1.2}
\setlength{\parskip}{.5\baselineskip}

\begin{document}
\maketitle

\subsection*{Topic}

In this IDP project we will investigate Starlink internals. Starlink is a project from SpaceX aiming to provide fast internet connection
anywhere in the world. We aim to understand in detail internal routing decision, general behavior for handling incoming packets and
understand how satellite handovers are performed.
We plan to fuzz the dish internal GRPC dishy API to see if we manage to get access to internal statistics.
Additional information will be extracted by dissecting the Android application provided to manage the dish.
Eventually we will develop some tooling to interact with the API, in order to abstract some of the logic in the event 
someone wants to perform some measurements, additional tooling may be developed to understand how satellite handovers are made. 
We will also try to understand the Starlink network behavior in order to try to make sense of sudden ping or bandwidth changes when doing measurements.
\subsection*{Related Work}

There is not much literature about internals of the dish and about the API, there's an interesting hardware-level hacking talk given at DEFCON.

We found some research regarding measurements \cite{10.1145/3517745.3561416} \cite{10.1145/3517745.3561457}

\begin{researchquestions}
\item Are there any hidden API endpoints?
\item Can the dish data be used to deduct internal starlink network behavior?
\item Can we extract some data that helps us verify the data the API is offering?
\end{researchquestions}

\subsection*{Approach}
In the first part we will try to understand and document the API, then, we will work on a library to make access to the API easier.
We will develop this in Python3, because of the great support of other libraries and because it seems to be the de-facto standard language
for data analysis and measurements. This will simplify data gathering for other researchers willing to do some more measurements.

We will intercept API calls made by the Android application using a proxy like Burpsuite Community Edition, and will eventually dissect the APK file
using tools like Jadx or Apktool. If needed we can also use Frida to do some dynamic patching.


\subsection*{Additional Hardware Requirements}
\begin{itemize}
  \item Starlink terminal (router and dish)
  \item a rooted Android phone to dissect the application
\end{itemize} 



\subsection*{Planned Schedule}
  \begin{ganttchart}[
      x unit=0.7mm,
      y unit title=0.6cm,
      title height=1,
      title label font=\small,
      newline shortcut=true,
      bar label font=\scriptsize,
      milestone label font=\scriptsize,
      milestone inline label node/.append style={left=2mm},
      milestone/.append style={fill=orange, shape=rectangle},
      y unit chart=.6cm,
      time slot format=isodate,
      time slot unit=day,
      bar height=0.7,
    ]{2023-03-01}{2023-08-31}
    \gantttitlecalendar{month=name}\\
    \ganttbar{Explore and document dish API}{2023-03-01}{2023-03-31}\\
    \ganttbar{Dissect the Android application}{2023-04-01}{2023-04-31}\\
    \ganttbar{Develop  some tooling to interact with API }{2023-05-01}{2023-05-31}\\
    \ganttbar{Use the tool to understand internals}{2023-06-01}{2023-08-31}
  \end{ganttchart}
  \label{fig:work-plan}

  \subsection*{Lecture}
  To complement my IDP I chose "Analysis, Modeling and Simulation of Communication Networks", because I the topics covered (packet level simulation and topology models) will help me in understanding routing decisions taken by the dish. Moreover the practical approach to the arguments will be 
  useful for the implementation of solutions to test performance and behaviour of the dish.

\bibliographystyle{IEEEtran}
\scriptsize
\bibliography{IEEEabrv,lit}

\end{document}

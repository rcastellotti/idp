% intro talk in german
%\documentclass[NET,a4paper,12pt,ngerman]{netforms}

% intro talk in english
\documentclass[NET,a4paper,12pt,english]{netforms}

\usepackage[utf8]{inputenc}
\usepackage{tumlang}
\usepackage{tumcontact}
\usepackage{scrlayer-scrpage}
\usepackage[textsize=scriptsize]{todonotes}
% Disable todonotes with
%\setuptodonotes{disable}
\setuptodonotes{inline}
\usepackage{pgfgantt}
\usepackage{float}
\usepackage{enumitem}
\newlist{researchquestions}{enumerate}{1}
\setlist[researchquestions,1]{label=\textbf{\textit{RQ \arabic*}}, resume}


\geometry{%
	top=20mm,
	bottom=20mm,
	left=25mm,
	right=25mm,
	headsep=1.5cm,
	includehead,
}

% Alle Konfigurationsbefehle sind optional. Fehlende Befehle fueheren einfach
% zu "blank forms".

% Type of thesis
% valid parameters: bachelor,master,diplom,idp,gr,hiwi,other
% If 'other' is chosen an optional parameter can be passed along (\type[optional]{other}).
% Without an optional parameter 'other' is chosen as a description.
\type{idp}

% \studiengang{} details the field of study
% valid parameters:
%   "Informatik",
%   "Wirtschaftsinformatik",
%   "Robotics, Cognition, Intelligence"
%   "Informatik: Games Engineering"
\studiengang{Informatik}

% Informationen ueber den Studenten. Sollte selbsterklaerend sein.
\anrede{Herr}
\nachname{Castellotti}
\vorname{Roberto}
\matrikel{03767095}
\rbgaccount{casr}
\semester{2}{SoSe\,2023}
\studientelefon{}{0039379181686}
\heimattelefon{}{--}
\studienadresse{Kantsraße}{17 München}
\heimatadresse[adresszusatz=,appartment=]{Kantsraße}{17}
\mail{r.castellotti@tum.de}

% Informationen ueber die Arbeit. Sollte selbsterklaerend sein.
\themensteller{\chairhead}
\beginn{03}{2023}
\endt{08}{2023}
\betreuer{Leander Seidlitz, Johannes Zirngibl}
\title{Development of a Framework for Retrieval of Parameters of the Starlink Dishy}{Entwurf eines Frameworks zur Informationsgewinnung von Parametern der Starlink Dishy}

% Nur für Bachelorarbeiten
\sprache{englisch} % options: deutsch | englisch

% Falls \type{hiwi} gesetzt wurde, wird die Taetigkeit auf dem Aufnahmeformular
% des Lehrstuhls angegeben.
\taetigkeit{test}



\pagestyle{scrheadings}
\clearscrheadfoot
\chead{\TUMheader{1cm}}

\renewcommand{\maketitle}{%
	\begin{center}
		\textbf{\introductoryheadline}%

		\Large%
		\textbf{\thetitle}%
	\end{center}

	\footnotesize%
	\hrule
	\vskip1ex
	\begin{tabular}{ll}
		\thenamelabel: & \thevorname{} \textbf{\thenachname}\\
		\theadvisorlabel: & \hspace*{-.5ex}\thebetreuer\\
		\thesupervisorlabel: & \chairhead\\
		\thebeginlabel: & \thebeginnmonat/\thebeginnjahr\\
		\theendlabel: & \theendmonat/\theendjahr\\
	\end{tabular}
	\vskip1ex
	\hrule
	\vskip4ex
}

\linespread{1.2}
\setlength{\parskip}{.5\baselineskip}

\begin{document}
\maketitle

\subsection*{Topic}

\todo{Add Motivation: Why is this topic relevant? Which ``big'' problem does it solve?}
This Bachelor's Thesis shall implement a wide band scanning application
resembling a spectrum analyzer using software defined radio hardware.
It shall be possible to both sweep the whole spectrum supported by the hardware
in use as well as tuning into specific parts of that spectrum.
In addition it shall be possible to record a narrow band of the spectrum for
both offline-analyses and replay at a later time.
The intended field of application is, for instance, the generation of
reproducible error signals in a testbed or empirically measuring the gain
of different antennas.

\subsection*{Related Work}

\todo{Explain here how this project compares to existing publications}

The thesis is thematically similar to the Master's Thesis \emph{Automatic Line
Code Detection of RF Signals on Software Defined Radios}, which also implemented a spectrum analyzer.
However, it was never planned to record or replay signals.
In addition, this thesis lacks some scientific rigor with respect to the
validity of recorded samples.


\subsection*{Research Questions}

\todo{Optional section: Maybe you can formulate research questions which you try to answer in this thesis}

\begin{researchquestions}
\item Do USB interfaces interfere with SDR?
\end{researchquestions}

\subsection*{Approach}
In the first part we will make measurements and statistic evaluations to
ensure, that the acquired data, which comes from the SDR, is valid.  Therefore
we will test, if the samples are valid right after the internal tuning phase of
the SDR.
We will also investigate whether or not external influences such as the USB
interface interfere with the process.
Other influence such as the quantization error will also be analyzed.
The latter are of particular interest when it comes to replay of previously
recorded signals.
The second part is about the scanning application itself.
The program can use the wide range of the present USRP hardware, which covers a
frequency range from 75\,MHz to 6\,GHz.
The tool would scan the whole band for online investigation or store the
samples for later use.
Further more, you will be able to rescan narrow bands of the spectrum to have a
closer look whats happening there.
The UI will consist of a zoomable heatmap (waterfall diagram) interface with
mouse support.

To get proper values to display, we have to consider different parameters like
integration time and gain settings etc.
These parameters will also be part of the research.
The software will be based on the GNUradio software defined radio
framework.  The additional work is planed to be done with python3.
The used hardware consists of 2 URSP b210, where for the main part only one is
needed.

\subsection*{Skills}
\todo{Optional section: include only if you
have worked on projects that are relevant for this thesis.}

I did some previous work with SDR hardware, the RTL-SDR and the
hackrf.
For this work the packet generator MoonGen was used~\cite{DBLP:conf/imc/EmmerichGRWC15}.
I am quite familiar with the software (GNUradio) and
hardware environment.

\subsection*{Additional Hardware Requirements}
\todo{Optional section: include only if you
    need new hardware for this thesis.}

\subsection*{Planned Schedule}
\begin{figure}[H]
  \centering
  \begin{ganttchart}[
      x unit=1.25mm,
      y unit title=0.6cm,
      title height=1,
      title label font=\small,
      bar label font=\scriptsize,
      milestone label font=\scriptsize,
      milestone inline label node/.append style={left=2mm},
      milestone/.append style={fill=orange, shape=rectangle},
      y unit chart=.6cm,
      time slot format=isodate,
      time slot unit=day,
      inline,
      bar height=0.7,
    ]{2016-04-15}{2016-08-25}
    \gantttitlecalendar{month=name}\\
    \ganttbar{Reading}{2016-04-15}{2016-04-28}\\
    \ganttbar{Setup testbed}{2016-04-15}{2016-05-05}\\
    \ganttbar{Work on first prototype}{2016-05-01}{2016-05-30}\\
    \ganttbar{Perform measurements}{2016-05-20}{2016-07-15}\\
    \ganttmilestone{Intermediate talk}{2016-06-15}\\
    \ganttbar{Write thesis}{2016-06-30}{2016-08-15}\\
    \ganttmilestone{Submission}{2016-08-14}
  \end{ganttchart}
  \label{fig:work-plan}
\end{figure}

\bibliographystyle{IEEEtran}
\scriptsize
\bibliography{IEEEabrv,lit}

\end{document}

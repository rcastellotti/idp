\documentclass[NET,english,beameralt]{tumbeamer}
% \usepackage[options]{hyperref}
\usepackage{listings}
\usepackage[utf8]{inputenc}
\usepackage{packages}
\usepackage{beamermods}

\author[Roberto Castellotti]{Roberto Castellotti}
\title[IDP]{Development of a Framework for Retrieval \\ of Parameters of the Starlink Dish}
\advisor{Leander Seidlitz, Johannes Zirngibl}
\thesistype{final}{idp}
\usepackage{pgfpages}
\usepackage{ifthen}
% ============================================================================
% jobname solution
% ============================================================================

\newdate{date}{13}{12}{2023}
\date{\displaydate{date}}


\newif\ifsolution%
\ifthenelse{\equal{\detokenize{notes}}{\jobname}}{%
\setbeameroption{show notes on second screen=bottom}
\setbeamercolor{note page}{bg=white, fg=black}
\setbeamercolor{note title}{bg=white!95!black, fg=black}
}{
}

% TeXLive 2018 compatibility: https://tex.stackexchange.com/questions/426088/texlive-pretest-2018-beamer-and-subfig-collide
\makeatletter
\let\@@magyar@captionfix\relax
\makeatother


\begin{document}

\begin{frame}{Starlink 101}
\begin{figure}
    \includegraphics[width=0.75\textwidth]{pics/starlink-101.png}
    \caption[short]{starlink in a nutshell (ignoring ISL), from \cite{izhikevich2023democratizing}}
\end{figure}
\end{frame}

\begin{frame}{Understanding routing decisions}
\begin{itemize}
    \item retrieved ip address blocks from major cloud providers (aws,azure,oracle), as we know their position
    \footnote[]{the fact we know the position doesn't really mean a traceroute to a certain address is really a
    traceroute to that geographic area}
    \item chose 5 geographically sparse targets around the globe (for aws: ap-northeast-2, us-east-1, ap-south-1,
    sa-east-1, me-south-1 )
    \item tracerouted the targets over several days 
\end{itemize}
\end{frame}

\begin{frame}{Understanding routing decisions}
\begin{figure}
    \includegraphics[width=0.75\textwidth]{pics/aws_7_icmp.png}
    \caption[short]{First 7 hops of traceroutes to 5 AWS datacenters using ICMP}
\end{figure}
\end{frame}

\begin{frame}{Visualize visible satellites}
\begin{itemize}
    \item from \href{celestrak.org}{celestrak.org} we can download a list of Starlink's satellites TLEs
    \item A two-line element set (TLE) is a data format encoding a list of orbital elements of an Earth-orbiting object
    for a given point in time, the epoch. Using a suitable prediction formula, the state (position and velocity) at any
    point in the past or future can be estimated to some accuracy. (from wikipedia.org)
\end{itemize}
\end{frame}

\begin{frame}[fragile]{\texttt{common.calculate\_visible\_satellites}}
    \begin{lstlisting}[
        language=python,
        caption={visualizing a single obstruction map},
        captionpos=b,
        label=listing-obs,
        basicstyle=\small,
    ]
def calculate_visible_satellites(...):
# ...
satellites = load.tle_file(stations_url)
observer = Topos(observer_latitude, observer_longitude, observer_elevation)
t = load.timescale().now()

# Calculate satellite positions
positions = [(sat, (sat - observer).at(t)) for sat in satellites]

# Filter visible satellites
visible_satellites = []
for sat, position in positions:
    alt, az, distance = position.altaz()
    # Satellite is above the horizon
    if alt.degrees > 0 and distance.km < distance_km:
        visible_satellites.append((sat, alt, az))

return visible_satellites
    \end{lstlisting}
\end{frame}

\begin{frame}{Visualizing patterns in visible satellites}
    \begin{figure}
        \includegraphics[width=1\textwidth]{pics/visualizing-how-long-satellites-are-visible-for.png}
        \caption[short]{Visualizing patterns in visible satellites}
    \end{figure}
\end{frame}

\begin{frame}{The gRPC api}
    \begin{itemize}
        \item the dish exposes a gRPC api with server reflection, "runtime construction of requests without having stub
        information precompiled into the client."
        \footnote{\href{https://github.com/grpc/grpc/blob/master/doc/server-reflection.md}{https://github.com/grpc/grpc/blob/master/doc/server-reflection.md}}
        \item 55 "methods" are available, most of them don't work, we have 2 categories of errors: \texttt{Uninmplemented},
        \texttt{PermissionDenied} and a couple of some other specific errors 
        \item working methods: \texttt{reboot}, \texttt{get\_status}, \texttt{start\_dish\_self\_test}, \texttt{get\_history}, \texttt{get\_device\_info}, \texttt{dish\_power\_save}, \texttt{dish\_get\_config}, \texttt{get\_obstruction\_map}
        \item to see all methods:
        \href{https://gist.github.com/rcastellotti/e20630366dfeaeada6cc2680f562f6ac}{https://gist.github.com/rcastellotti/e20630366dfeaeada6cc2680f562f6ac}
    \end{itemize}
\end{frame}

\begin{frame}[fragile]{Querying the  \texttt{dish\_get\_obstruction\_map} endpoint }
    \begin{lstlisting}[
        basicstyle=\footnotesize,
    ]
"apiVersion":"9",
"dishGetObstructionMap":{
    "minElevationDeg":10.0,
    "numCols":123,
    "numRows":123,
    "snr":[
        -1.0,
        -1.0,
        1.0,
        1.0,
        -1.0,
        -1.0,
        "...",
        1.0,
        1.0,
        1.0,
        -1.0,
        -1.0,
        -1.0
    ]
}
    \end{lstlisting}
\end{frame}

\begin{frame}[fragile]{Visualizing a single Obstruction Map}
    \begin{lstlisting}[
        caption={Visualizing a single obstruction map},
        captionpos=b,
        basicstyle=\small,
    ]
    import json
    import numpy as np
    import matplotlib.pyplot as plt

    f = "1692089163.json"
    map = json.load(open(f))
    map = map["dishGetObstructionMap"]["snr"]
    map = np.array(map).reshape(123, 123)
    plt.imshow(map)
    plt.show()
    \end{lstlisting}
\end{frame}

\begin{frame}[fragile]
    \begin{figure}
        \includegraphics[width=0.4\columnwidth]{pics/map1.png}
    \end{figure}
\end{frame}

\begin{frame}[fragile]
    \begin{figure}
        \includegraphics[width=0.4\columnwidth]{pics/map2.png}
    \end{figure}
\end{frame}

\begin{frame}[fragile]
    \begin{figure}
        \includegraphics[width=0.4\columnwidth]{pics/map3.png}
    \end{figure}
\end{frame}

\begin{frame}[fragile]
    \begin{figure}
        \includegraphics[width=0.4\columnwidth]{pics/map4.png}
    \end{figure}
\end{frame}


\begin{frame}{Obstruction Maps as Matrices (no handover)}
    $\begin{bmatrix}
        -1 & -1 & \color{red}1 &           -1 & -1 \\
        -1 & -1 &           -1 & \color{red}1 & -1 \\
        -1 & -1 &           -1 &           -1 & -1 \\
        -1 & -1 &           -1 &           -1 & -1 \\
        -1 & -1 &           -1 &           -1 & -1 \\ 
        \end{bmatrix}
        +
        \begin{bmatrix}
        -1 & -1 & \color{red}1 &           -1 &           -1 \\
        -1 & -1 &           -1 & \color{red}1 &           -1 \\
        -1 & -1 &           -1 &           -1 & \color{red}1 \\
        -1 & -1 &           -1 &           -1 &           -1 \\
        -1 & -1 &           -1 &           -1 &           -1 \\
        \end{bmatrix}
        =
        \begin{bmatrix}
        -2 & -2 & 2 &  -2 &           -2 \\
        -2 & -2 & -2 &  2 &           -2 \\
        -2 & -2 & -2 & -2 & \color{red}0 \\
        -2 & -2 & -2 & -2 &            -2 \\
        -2 & -2 & -2 & -2 &            -2 \\
    \end{bmatrix}$

    \vspace{5mm}
    
\end{frame}

\begin{frame}{Obstruction Maps as Matrices (handover)}
    $\begin{bmatrix}
        -1 & -1 & \color{red}1 &           -1 & -1 \\
        -1 & -1 &           -1 & \color{red}1 & -1 \\
        -1 & -1 &           -1 &           -1 & -1 \\
        -1 & -1 &           -1 &           -1 & -1 \\
        -1 & -1 &           -1 &           -1 & -1 \\
    \end{bmatrix}
    +
    \begin{bmatrix}
        -1 & -1 & \color{red}1 &           -1 & -1 \\
        -1 & -1 &           -1 & \color{red}1 & -1 \\
        -1 & -1 &           -1 &           -1 & -1 \\
        -1 & -1 &           -1 &           -1 & -1 \\
        1 & -1 &            -1 &           -1 & -1 \\
    \end{bmatrix}
    =
    \begin{bmatrix}
        -2 & -2 & 2 & -2 & -2 \\
        -2 & -2 & -2 & 2 & -2 \\
        -2 & -2 & -2 & -2 & -2 \\
        -2 & -2 & -2 & -2 & -2 \\
        \color{red}0 & -2 & -2 & -2 & -2 \\
    \end{bmatrix}$
\end{frame}

\begin{frame}[fragile]{Correlation between satellite handovers bandwidth drops}
    \begin{figure}
        \includegraphics[width=0.8\columnwidth]{pics/correlation_handovers_bw.png}
    \end{figure}
\end{frame}

\section{Bibliography}
\begin{frame}[allowframebreaks]
    \bibliographystyle{abbrv}
    \setbeamertemplate{bibliography item}[text]
    \footnotesize
    \bibliography{lit}
\end{frame}

\end{document}
